
%% bare_conf.tex
%% V1.4b
%% 2015/08/26
%% by Michael Shell
%% See:
%% http://www.michaelshell.org/
%% for current contact information.
%%
%% This is a skeleton file demonstrating the use of IEEEtran.cls
%% (requires IEEEtran.cls version 1.8b or later) with an IEEE
%% conference paper.
%%
%% Support sites:
%% http://www.michaelshell.org/tex/ieeetran/
%% http://www.ctan.org/pkg/ieeetran
%% and
%% http://www.ieee.org/

%%*************************************************************************
%% Legal Notice:
%% This code is offered as-is without any warranty either expressed or
%% implied; without even the implied warranty of MERCHANTABILITY or
%% FITNESS FOR A PARTICULAR PURPOSE! 
%% User assumes all risk.
%% In no event shall the IEEE or any contributor to this code be liable for
%% any damages or losses, including, but not limited to, incidental,
%% consequential, or any other damages, resulting from the use or misuse
%% of any information contained here.
%%
%% All comments are the opinions of their respective authors and are not
%% necessarily endorsed by the IEEE.
%%
%% This work is distributed under the LaTeX Project Public License (LPPL)
%% ( http://www.latex-project.org/ ) version 1.3, and may be freely used,
%% distributed and modified. A copy of the LPPL, version 1.3, is included
%% in the base LaTeX documentation of all distributions of LaTeX released
%% 2003/12/01 or later.
%% Retain all contribution notices and credits.
%% ** Modified files should be clearly indicated as such, including  **
%% ** renaming them and changing author support contact information. **
%%*************************************************************************


% *** Authors should verify (and, if needed, correct) their LaTeX system  ***
% *** with the testflow diagnostic prior to trusting their LaTeX platform ***
% *** with production work. The IEEE's font choices and paper sizes can   ***
% *** trigger bugs that do not appear when using other class files.       ***                          ***
% The testflow support page is at:
% http://www.michaelshell.org/tex/testflow/



\documentclass[conference]{IEEEtran}
\usepackage{listings}
% Some Computer Society conferences also require the compsoc mode option,
% but others use the standard conference format.
%
% If IEEEtran.cls has not been installed into the LaTeX system files,
% manually specify the path to it like:
% \documentclass[conference]{../sty/IEEEtran}





% Some very useful LaTeX packages include:
% (uncomment the ones you want to load)


% *** MISC UTILITY PACKAGES ***
%
%\usepackage{ifpdf}
% Heiko Oberdiek's ifpdf.sty is very useful if you need conditional
% compilation based on whether the output is pdf or dvi.
% usage:
% \ifpdf
%   % pdf code
% \else
%   % dvi code
% \fi
% The latest version of ifpdf.sty can be obtained from:
% http://www.ctan.org/pkg/ifpdf
% Also, note that IEEEtran.cls V1.7 and later provides a builtin
% \ifCLASSINFOpdf conditional that works the same way.
% When switching from latex to pdflatex and vice-versa, the compiler may
% have to be run twice to clear warning/error messages.






% *** CITATION PACKAGES ***
%
%\usepackage{cite}
% cite.sty was written by Donald Arseneau
% V1.6 and later of IEEEtran pre-defines the format of the cite.sty package
% \cite{} output to follow that of the IEEE. Loading the cite package will
% result in citation numbers being automatically sorted and properly
% "compressed/ranged". e.g., [1], [9], [2], [7], [5], [6] without using
% cite.sty will become [1], [2], [5]--[7], [9] using cite.sty. cite.sty's
% \cite will automatically add leading space, if needed. Use cite.sty's
% noadjust option (cite.sty V3.8 and later) if you want to turn this off
% such as if a citation ever needs to be enclosed in parenthesis.
% cite.sty is already installed on most LaTeX systems. Be sure and use
% version 5.0 (2009-03-20) and later if using hyperref.sty.
% The latest version can be obtained at:
% http://www.ctan.org/pkg/cite
% The documentation is contained in the cite.sty file itself.






% *** GRAPHICS RELATED PACKAGES ***
%
\ifCLASSINFOpdf
\usepackage[pdftex]{graphicx}
  % declare the path(s) where your graphic files are
\graphicspath{{./img/}}
  % and their extensions so you won't have to specify these with
  % every instance of \includegraphics
\DeclareGraphicsExtensions{.pdf,.jpeg,.png}
\else
  % or other class option (dvipsone, dvipdf, if not using dvips). graphicx
  % will default to the driver specified in the system graphics.cfg if no
  % driver is specified.
  % \usepackage[dvips]{graphicx}
  % declare the path(s) where your graphic files are
  % \graphicspath{{../eps/}}
  % and their extensions so you won't have to specify these with
  % every instance of \includegraphics
  % \DeclareGraphicsExtensions{.eps}
\fi
% graphicx was written by David Carlisle and Sebastian Rahtz. It is
% required if you want graphics, photos, etc. graphicx.sty is already
% installed on most LaTeX systems. The latest version and documentation
% can be obtained at: 
% http://www.ctan.org/pkg/graphicx
% Another good source of documentation is "Using Imported Graphics in
% LaTeX2e" by Keith Reckdahl which can be found at:
% http://www.ctan.org/pkg/epslatex
%
% latex, and pdflatex in dvi mode, support graphics in encapsulated
% postscript (.eps) format. pdflatex in pdf mode supports graphics
% in .pdf, .jpeg, .png and .mps (metapost) formats. Users should ensure
% that all non-photo figures use a vector format (.eps, .pdf, .mps) and
% not a bitmapped formats (.jpeg, .png). The IEEE frowns on bitmapped formats
% which can result in "jaggedy"/blurry rendering of lines and letters as
% well as large increases in file sizes.
%
% You can find documentation about the pdfTeX application at:
% http://www.tug.org/applications/pdftex





% *** MATH PACKAGES ***
%
% \usepackage{amsthm}
% \usepackage{amsmath}
% A popular package from the American Mathematical Society that provides
% many useful and powerful commands for dealing with mathematics.
%
% Note that the amsmath package sets \interdisplaylinepenalty to 10000
% thus preventing page breaks from occurring within multiline equations. Use:
%\interdisplaylinepenalty=2500
% after loading amsmath to restore such page breaks as IEEEtran.cls normally
% does. amsmath.sty is already installed on most LaTeX systems. The latest
% version and documentation can be obtained at:
% http://www.ctan.org/pkg/amsmath





% *** SPECIALIZED LIST PACKAGES ***
%
%\usepackage{algorithmic}
% algorithmic.sty was written by Peter Williams and Rogerio Brito.
% This package provides an algorithmic environment fo describing algorithms.
% You can use the algorithmic environment in-text or within a figure
% environment to provide for a floating algorithm. Do NOT use the algorithm
% floating environment provided by algorithm.sty (by the same authors) or
% algorithm2e.sty (by Christophe Fiorio) as the IEEE does not use dedicated
% algorithm float types and packages that provide these will not provide
% correct IEEE style captions. The latest version and documentation of
% algorithmic.sty can be obtained at:
% http://www.ctan.org/pkg/algorithms
% Also of interest may be the (relatively newer and more customizable)
% algorithmicx.sty package by Szasz Janos:
% http://www.ctan.org/pkg/algorithmicx

\usepackage{amsthm, amsmath,amssymb,amsbsy,amsfonts,amstext,rotating}
\usepackage{alltt,multicol,pifont}
\usepackage{booktabs,multirow,subfigure,epsfig}
\usepackage{xspace}
\newcommand{\eplopmark}{\ding{73}}

% *** ALIGNMENT PACKAGES ***
%
\usepackage{array,tabularx}
% Frank Mittelbach's and David Carlisle's array.sty patches and improves
% the standard LaTeX2e array and tabular environments to provide better
% appearance and additional user controls. As the default LaTeX2e table
% generation code is lacking to the point of almost being broken with
% respect to the quality of the end results, all users are strongly
% advised to use an enhanced (at the very least that provided by array.sty)
% set of table tools. array.sty is already installed on most systems. The
% latest version and documentation can be obtained at:
% http://www.ctan.org/pkg/array


% IEEEtran contains the IEEEeqnarray family of commands that can be used to
% generate multiline equations as well as matrices, tables, etc., of high
% quality.




% *** SUBFIGURE PACKAGES ***
%\ifCLASSOPTIONcompsoc
%  \usepackage[caption=false,font=normalsize,labelfont=sf,textfont=sf]{subfig}
%\else
%  \usepackage[caption=false,font=footnotesize]{subfig}
%\fi
% subfig.sty, written by Steven Douglas Cochran, is the modern replacement
% for subfigure.sty, the latter of which is no longer maintained and is
% incompatible with some LaTeX packages including fixltx2e. However,
% subfig.sty requires and automatically loads Axel Sommerfeldt's caption.sty
% which will override IEEEtran.cls' handling of captions and this will result
% in non-IEEE style figure/table captions. To prevent this problem, be sure
% and invoke subfig.sty's "caption=false" package option (available since
% subfig.sty version 1.3, 2005/06/28) as this is will preserve IEEEtran.cls
% handling of captions.
% Note that the Computer Society format requires a larger sans serif font
% than the serif footnote size font used in traditional IEEE formatting
% and thus the need to invoke different subfig.sty package options depending
% on whether compsoc mode has been enabled.
%
% The latest version and documentation of subfig.sty can be obtained at:
% http://www.ctan.org/pkg/subfig




% *** FLOAT PACKAGES ***
%
%\usepackage{fixltx2e}
% fixltx2e, the successor to the earlier fix2col.sty, was written by
% Frank Mittelbach and David Carlisle. This package corrects a few problems
% in the LaTeX2e kernel, the most notable of which is that in current
% LaTeX2e releases, the ordering of single and double column floats is not
% guaranteed to be preserved. Thus, an unpatched LaTeX2e can allow a
% single column figure to be placed prior to an earlier double column
% figure.
% Be aware that LaTeX2e kernels dated 2015 and later have fixltx2e.sty's
% corrections already built into the system in which case a warning will
% be issued if an attempt is made to load fixltx2e.sty as it is no longer
% needed.
% The latest version and documentation can be found at:
% http://www.ctan.org/pkg/fixltx2e


%\usepackage{stfloats}
% stfloats.sty was written by Sigitas Tolusis. This package gives LaTeX2e
% the ability to do double column floats at the bottom of the page as well
% as the top. (e.g., "\begin{figure*}[!b]" is not normally possible in
% LaTeX2e). It also provides a command:
%\fnbelowfloat
% to enable the placement of footnotes below bottom floats (the standard
% LaTeX2e kernel puts them above bottom floats). This is an invasive package
% which rewrites many portions of the LaTeX2e float routines. It may not work
% with other packages that modify the LaTeX2e float routines. The latest
% version and documentation can be obtained at:
% http://www.ctan.org/pkg/stfloats
% Do not use the stfloats baselinefloat ability as the IEEE does not allow
% \baselineskip to stretch. Authors submitting work to the IEEE should note
% that the IEEE rarely uses double column equations and that authors should try
% to avoid such use. Do not be tempted to use the cuted.sty or midfloat.sty
% packages (also by Sigitas Tolusis) as the IEEE does not format its papers in
% such ways.
% Do not attempt to use stfloats with fixltx2e as they are incompatible.
% Instead, use Morten Hogholm'a dblfloatfix which combines the features
% of both fixltx2e and stfloats:
%
% \usepackage{dblfloatfix}
% The latest version can be found at:
% http://www.ctan.org/pkg/dblfloatfix




% *** PDF, URL AND HYPERLINK PACKAGES ***
%
\usepackage{url}
% url.sty was written by Donald Arseneau. It provides better support for
% handling and breaking URLs. url.sty is already installed on most LaTeX
% systems. The latest version and documentation can be obtained at:
% http://www.ctan.org/pkg/url
% Basically, \url{my_url_here}.




% *** Do not adjust lengths that control margins, column widths, etc. ***
% *** Do not use packages that alter fonts (such as pslatex).         ***
% There should be no need to do such things with IEEEtran.cls V1.6 and later.
% (Unless specifically asked to do so by the journal or conference you plan
% to submit to, of course. )

\usepackage[utf8]{inputenc}
\usepackage[english]{babel}

\theoremstyle{definition}
\newtheorem{definition}{Definition}

% correct bad hyphenation here
\hyphenation{op-tical net-works semi-conduc-tor}

\usepackage{tipa}
\usepackage{array,tabularx}
\usepackage{color}
\definecolor{applegreen}{rgb}{0.55, 0.71, 0.0}
\definecolor{bananayellow}{rgb}{1.0, 0.88, 0.21}
\definecolor{blue}{rgb}{0.0, 0.53, 0.74}

\begin{document}
%
% paper title
% Titles are generally capitalized except for words such as a, an, and, as,
% at, but, by, for, in, nor, of, on, or, the, to and up, which are usually
% not capitalized unless they are the first or last word of the title.
% Linebreaks \\ can be used within to get better formatting as desired.
% Do not put math or special symbols in the title.
\title{Test event generation for a fall-detection IoT system}


% author names and affiliations
% use a multiple column layout for up to three different
% affiliations
\author{\IEEEauthorblockN{Lorena Guti\'errez-Madro\~nal, Inmaculada Medina-Bulo}
\IEEEauthorblockA{UCASE Research Group\\
University of Cadiz, Spain\\
Email: \{lorena.gutierrez, inmaculada.medina\}@uca.es}
\and
\IEEEauthorblockN{Luigi La Blunda, Matthias F. Wagner}
\IEEEauthorblockA{WSN and IOT Research Group\\
Frankfurt University of Applied Sciences\\
Email: \{l.lablunda,mfwagner\}@fb2.fra-uas.de}}

\maketitle

\begin{abstract}
The Internet of Things (IoT) is a very popular paradigm which has been applied to different areas 
such as smart cities, medicine, business process, etc. The IoT system main inconvenience is to make
decisions in real time according to a huge amount of information that arrives as events. This information
is filtered thanks to the Event Processing Languages (EPL), which use patterns to define the relevant situations.
So, given that filter the correct information is crucial to carry out the established actions, testing these
IoT systems is imperative.
In the majority of the relevant situations to detect, the events have a specific behaviour which must
be simulated to test the IoT system. Moreover, in several situations is quite difficult to obtain test events with
specific values: adverse environment conditions, rise or fall in blood pressure, heart attack, falls... 
In this paper we introduce a fall analysis and the test event generation using IoT-TEG tool. The fall analysis
has highlighted the special behaviour of the fall-involved events and the necessity to improve the IoT-TEG with
a new functionality which allows to define the desired behaviour by defining behaviour rules.
\end{abstract}

\IEEEpeerreviewmaketitle

\section{Introduction}

Due to the progress of health care, the longevity of people increases and this leads to an ageing society. Elderly people are exposed 
to a higher risk of falling because of the growing age and multiple diseases, which cause serious injuries that require long convalescence 
and restriction of mobility. According to the survey of the \textit{Robert Koch Institute}~\cite{Varnaccia2013}, 53.7\% of accidents in the age 
group over 60 are caused by falls. Statistically, about one-third of the elderly people suffer severe lesions and the half of them suffer 
fall-events repeatedly~\cite{Schott2008}. Falls are not deployed by a single cause, 90\% of them occurred from multiple factors. These 
factors refer to old-age or illness (intrinsic factors) or external factors e.g. hazards which occur at home, in traffic or during activities 
of daily life (extrinsic factors)~\cite{Schott2008}. The founder of \textit{Vigilio Telemedical} reported that yearly more than 20 million 
elderly over the age of 65 in Europe experience fall-situations, that lead to traumatic based cases of death~\cite{Vigilio,APAOTS2013}. 
Additionally, people affected by Dementia and Parkinson have a higher risk to fall. In accordance with~\cite{Monks} research proved that 
Dementia-patients have a 20 times higher risk and Parkinson-patients a 10 times higher risk of falling than healthy people of the same age. 
To counteract these life-threatening situations a fast and fully automated assistance is needed, because an unconscious person may not be 
able to call the emergency services. An approach could be continuous monitoring of medical and/or physical signals via a wearable sensor 
network (see Figure~\ref{fig:simulation}). 

\begin{figure}[!h]
  \centering
  \includegraphics[scale=0.2]{img/Figure1}
  \caption[Escalation scheme]{Scheme fall simulation.}
  \label{fig:simulation}
\end{figure}

A prototype in form of a belt was developed, which is worn on the hip by the patient and consists 
of a five sensor nodes Body Area Network (BAN)~\cite{LaBlunda.2016,LaBlunda.2016b}. The obtained data are used to define a EPL of 
EsperTech~\cite{Esper:2016} pattern to detect falls. The first step of this prototype is to identify a fall from a fast movement, a siting
move or lying move, but the main goal of the to be finished system is to prevent the falls and act to prevent them. Given that to test 
this system is crucial, test events which simulate falls are necessary. In the literature can be found different type of falls, and it is
necessary to identify all of them in order to tell them apart from a no-fall; in this
study two type of falls will be analysed. 

IoT-TEG~\cite{TesisGutierrez2017,Gutierrez2017} is a tool which 
automatically generates test events of many type. Thanks to the obtained data from the sensors we have checked that the measured 
parameter during a fall, the acceleration, has an specific behaviour. As a consequence, the test events must be generated according to 
its behaviour. This problem is solved with the new functionality that IoT-TEG includes which is introduced in this paper. 
Moreover, the ongoing fall detection prototype will be analysed and its improvements will be described; the new functionality of 
IoT-TEG can be adapted according to the improvements of the fall detection prototype. The main contributions of this paper are:

\begin{itemize}
 \item \textbf{An analysis of the involved parameter in a fall}; while a person is falling, the acceleration is the parameter that can
 measure the movement of the body. This parameter is analysed in order to know its behaviour during two type of falls.
 \item \textbf{A new functionality of the IoT-TEG system} which allows to simulate the behaviour of different event attributes in order 
 to generate test events following a specific pattern.
 \item \textbf{A study of the fall detection prototype evolution}; the system which is been used has suffered some improvements. 
 We describe the evolution its architecture, how the data is analysed and the founded problems.
 \item \textbf{New definitions of two type of falls}; after the analysis of the acceleration during two type of falls a new definition
 for each one has been done. The obtained data of the fall detection prototype is used to define a EPL of EsperTech pattern to detect 
 those type of falls. 
\end{itemize}

The remained of this paper is organised as follows. Section~\ref{sec:relatedwork}
describes not only the related work of event generators, but also the existing
solutions for fall-detection. Section~\ref{sec:background} provides the basic
knowledge of falls, Event Processing Languages and IoT-TEG tool. The architecture
of the fall detection system, the fall analysis and some founded problems are 
introduced in Section~\ref{sec:basicprototype}. Section~\ref{sec:improvedprototype}
describes the improvements on the prototype, a new fall analysis and
a comparison of the obtained results. Finally, in Section~\ref{sec:conclusions}, 
we conclude our paper and make recommendations for future work.

\section{Related work}
\label{sec:relatedwork}

An overview about event generators reveals that the first event generators~\cite{dobbs2004houches,mangano2005tools} were focused on 
extremely specific topics such as environmental conditions for the simulation of high energy physics events at particle colliders. Nowadays,
the people and business need to control and monitor the things around them. The received information allows them
to make decisions and to act according to it. This is the reason of the creation of the IoT Platforms, which are the
key for the development of scalable IoT applications and services that connect objects, systems and people to each other.
However, not every IoT Platform is an IoT Platform~\cite{iot-analytics:2015}; for instance, some event generators that are integrated in an
enterprise software packages, which are increasingly allowing the integration of IoT devices, are often not advanced enough
to be classified as a full IoT Platform. Examples are given in the following lines:

The Timing System~\cite{Finland:2016} provides a complete timing distribution system including timing signal generation. Its event
generator is responsible for creating timing events which are sent out as serialised event frames.

The company Starcom~\cite{Starcom:2016} has developed an event generator to solve the problem of managing a huge number of
events. They state that their generator is capable of controlling the end event action, so the exact managers requirements can
be filtered. The tool is included in a kit distributed with their system.

The WebLogic Integration Solutions~\cite{WebLogic:2016} allows the managing and monitoring of entities and resources required for
WebLogic Integration applications. This system contains an event generator module which allows the creation and deployment of the 
event generators included as part of WebLogic Integration. The mentioned events generators allow to define event types
but they do not are capable to simulate a specific behaviour with a set of generated events. The relevant situations in
IoT systems are a sequence of activities with a determined behaviour; that is why IoT-TEG~\cite{TesisGutierrez2017,Gutierrez2017} 
includes this option.

Talking about fall-detection, there are several solutions that propose wearable sensors. A commercial solution which is 
available on the market is the VigiFall system~\cite{Vigilio,EuropeanCommission2013}, supported by the European Commission. This system 
includes a wearable self-adhesive accelerometer, several motion sensors which are fixed in the living area and a calling unit to provide 
a fully automated emergency call. The self-adhesive patch communicates with the infrared motion detectors and in the case of a fall the 
wearable patch sends a signal out. Thereby the infrared sensors placed in the room are capable to recognise that there are no movements anymore and 
contacts the central unit. When the central node receives this flag, the emergency call is executed. The weak point of VigiFall~\cite{Vigilio} is 
the system structure which depends on the sensor-based room infrastructure. Once the patient leaves this area the system is not capable 
to provide the functionalities in case of a fall-event. Igual et al.~\cite{Igual2013} examined different fall-detection approaches which they 
categorised into context-aware systems and wearable systems. Context-aware systems depend on the living area of the patient, where 
sensors and actuators placed in the environment interact with the wearable node of the person to detect falls. Another solution based 
on this principle is a video-based method, which facilitates a reliable detection of falls but the patients would be exposed to a 
loss of privacy and this is not well accepted. Additionally the high purchase price is a barrier for many people and the dependency 
on the environment makes these system useless. The other system-type analysed by Igual et al.~\cite{Igual2013} comprises wearable systems, which 
are worn on the body and based on a BAN. This solution is able to detect falls independent from the environment in contrast to 
context-aware systems. They depict wearable systems which use the sensor fusion principle of accelerometer and gyroscope 
and built-in solutions that represent the usage of integrated smartphone sensors.

Taking into consideration wearable fall-detection solutions the approach proposed by Li et al.~\cite{Li2009} will be explained subsequently, 
which was used for the development of our prototype. They introduce a fall-detection method based on a BAN that consists 
of two wearable sensor nodes. These two nodes comprise an accelerometer and gyroscope and are worn on the chest, node A, and thigh 
node B, (see Figure~\ref{fig:simulation}). The principle of this method differentiates between static postures and dynamic postures: 

\begin{itemize}
 \item Static postures: standing, sitting, lying and bending.
 \item Dynamic postures
 \begin{itemize}
  \item Activities of daily life: walking, walk on stairs, sit, jump, lay down and run.
  \item Fall-like motions: quick sit down upright and quick sit-down reclined.
  \item Flat surface falls: fall forward, fall backward, fall right and fall left.
  \item Inclined falls: fall on stairs.
 \end{itemize}
\end{itemize}

\begin{figure}[!h]
  \centering
  \includegraphics[scale=0.25]{img/BasePrototype.png}
  \caption[System architecture]{System architecture according to Li et al.~\cite{Li2009}}
  \label{fig:simulation}
\end{figure}

To decrease the computational effort of the micro-controller a three-phase algorithm was proposed, which is structured 
as follows:

\begin{enumerate}
 \item Phase activity analysis: check if person is in a static or dynamic position.
 \item Phase position analysis: if existing posture coincides with static posture, check whether the actual position corresponds to lying.
 \item Phase state transition analysis: if lying position, examine whether this transition was intentional or unintentional. The previous 5 
 seconds are used to analyse it. In the case that this position was unintentional, the system classifies it as a fall. 
\end{enumerate}

The weak point of this method is the differentiation between jumping into bed and falling against a wall with seated posture.

\section{Background}
\label{sec:background}

\subsection{Falls analysis}
\label{subsec:analysis}

The approach used to detect falls is based on the basic idea proposed by~\cite{Gjoreski2014,Kozina}. A typical acceleration pattern during 
a fall-event is a decreasing acceleration close to 0g (free fall), followed by an increasing acceleration value (see Figure~\ref{fig:grafica}). In a stationary position, 
the acceleration measured is around 1g (9.81$m/s^{2}$) and during falling around 0g (0$m/s^{2}$). Upon the impact to the ground (fall-event), 
the maximum acceleration (peak-value) is reached for a short time period and it is greater than 1g. This pattern can be used to 
detect fall-events using the integrated accelerometer of the sensor nodes (S1-S4). Important to know is that the acceleration 
magnitude which is illustrated in the Formula~\ref{eq:acel} was used for the fall-detection:

\begin{equation}\label{eq:acel}
 \overrightarrow{\left | \alpha \right |} = \sqrt{\alpha_{x}^{2} + \alpha_{y}^{2} + \alpha_{z}^{2}}
\end{equation}

\begin{figure}[!h]
  \centering
  \includegraphics[scale=0.8]{img/FallGraph}
  \caption[Acceleration during impact]{Acceleration during impact~\cite{Kozina}.}
  \label{fig:grafica}
\end{figure}

Taking into consideration the depicted Figure~\ref{fig:grafica}, we can apply the following rule to evaluate the incoming acceleration data:

\begin{equation}\label{eq:caida}
 \alpha_{max} - \alpha_{min} > 1g
\end{equation}

in 1 second window and $\alpha_{max}$ came after $\alpha_{min}$; where $\alpha_{max}$ is the maximum acceleration value, 
$\alpha_{min}$ is the minimum acceleration value and $1g\approx9.81m/s^{2}$. Looking at the rule an event is categorised as a fall when the 
difference between $\alpha_{max}$ and $\alpha_{min}$ is greater than 1g and $\alpha_{min}$ is followed by $\alpha_{max}$. 
Important is that this should happen within a time window of 1 second due to the fact, that a fall can occur in less than 1 second~\cite{Luder2009}.
To apply this rule we used the Event Processing Language (EPL) of EsperTech~\cite{Esper:2016} to integrate it in a IoT system which detect falls.

\subsection{Fall patterns with EPL of EsperTech}

Etzion and Niblett~\cite{Etz10} defined \textit{event processing agents} as software modules that process events. Such agents are specified using an 
event processing language, and there are a number of styles of event processing languages in use. The following styles are included:

\begin{itemize}
 \item Rule-oriented languages
 \begin{itemize}
 \item Production rules: Production rules are rules of the type \textit{if condition then action}: when the condition is satisfied, the action is performed.
 \item Active rules: Active rules are rules of the type \textit{event-condition-action}: when an event occurs, the conditions are evaluated and, if they are satisfied, 
 it triggers an action.
 \item Logic rules: A programming style based on logical assertions and deductive database. 
 \end{itemize}
 \item Imperative programming languages: The imperative programming languages define the operators which will be applied to the events. Each operator is a
 transformation in the event.
 \item Stream-oriented languages: The languages used to describe the queries are inspired by SQL and relational algebra, though not all of them are 
 based on SQL.
\end{itemize}

The EPL in which our work is based on is EPL of Esper~\cite{Esper:2016}, an EPL stream-oriented language. The main reasons of its selection are: 
it is an extension of SQL (world famous), it can be embedded into Java applications and it is open source. On the other hand, it is executed by Esper, a
CEP engine which can process around 500.000 events per second on a workstation, and between 70.000 and 200.000 events per second on a laptop (according to 
the company EsperTech).
 
As it was mentioned before, the EPL of EsperTech is a SQL like query language. However, unlike SQL that operates on tables, EPL operates on continuous stream of events. As a 
result, a row from a table in SQL is analogous to an event present in an event stream. An EPL statement starts executing continuously during 
runtime. While the execution is taking place, EPL queries will be triggered if the application receives pre-defined or timer triggering events.
 
 \renewcommand{\lstlistingname}{Example}
 
 \begin{lstlisting}[basicstyle=\ttfamily\footnotesize,language=SQL,caption=EPL of EsperTech query example,label=EPLqueries]
  select A as temp1, B as temp2 from 
    pattern [every temp1.temperature > 400 
    -> temp2.temperature > 400]
 \end{lstlisting}
 
In the above example (see Example~\ref{EPLqueries}) a ``Central'' needs to measure the temperature of its systems, its temperature gauges take a 
reading of the core temperature every second and send the data to a central monitoring system. The EPL of EsperTech query throws a warning if we have 
2 consecutive temperatures above a certain threshold (400). This is a situation where it is needed a quick reaction to emerging patterns in a 
stream of data events. A quick reaction is also needed in fall detection, the pattern which describes this situation will be shown in the 
corresponding section.
 
Because the difficulties to simulate falls, IoT-TEG is used in order to get automatically the fall test events. IoT-TEG can be adapted or
modified, if it is necessary, to generate any test events, even after the improvements in the fall-detection prototype.

\subsection{IoT test event generator}
\label{iotteg}

IoT-TEG is a Java-based tool which takes an event type definition file and a desired output
format (JSON, CSV, and XML, the most common across IoT platforms). IoT-TEG is made up of a
\emph{validator} and an \emph{event generator} (Figure~\ref{fig:IoT-EGArquitecture}). The
validator ensures the definition follows the rules set by IoT-TEG. The generator takes the
definition and generates the indicated number of events according to it.

\begin{figure}[!h]
  \centering
  \includegraphics[scale=0.65]{./img/IoT-EGArquitecture}
  \caption[IoT-TEG Architecture]{IoT-TEG Architecture.}
  \label{fig:IoT-EGArquitecture}
\end{figure}

Previous studies suggested there were no differences in testing effectiveness between using events
generated by IoT-TEG, or events recorded from various case studies~\cite{Gutierrez2017}. These
results confirm IoT-TEG can simulate many types of events occurring in industrial applications,
and solve the main challenges developers face when they test event-processing programs:
\begin{enumerate}
 \item Lack of data for testing,
 \item needing specific values for the events, and
 \item needing the source to generate the events.
\end{enumerate}

For the sake of clarity, Example~\ref{eventTypeDef} shows an event type
definition that could be used to test the queries of Example~\ref{EPLqueries}.

\begin{lstlisting}[basicstyle=\ttfamily\footnotesize,language=XML,caption=Event type definition example,label=eventTypeDef]
<?xml version="1.0" encoding="UTF-8"?>
<event_type name="TemperatureEvent">
  <block name="feeds" repeat="150">
    <field name="created_at" quotes="true" 
     type="ComplexType">
     <attribute type="Date" format="yy-MM-dd">
     </attribute>
     <attribute type="String" format="T"></attribute>
     <attribute type="Time" format="hh:mm">
     </attribute>
    </field>
    <field name="entry_id" quotes="false" 
     type="Integer" min="0" max="10000">
    </field>
    <field name="temperature" quotes="false" 
     type="Float" min="0" max="500" precision="1">
    </field>
  </block>
</event_type>
\end{lstlisting}

The defined event type contains three properties: \texttt{created\_at},
\texttt{entry\_id} and \texttt{temperature}. These properties are defined as
fields in the event type definition. The \texttt{created\_at} field is complex
type and \texttt{entry\_id} and \texttt{temperature} are simple types. The
property that is evaluated in the Example~\ref{eventTypeDef} queries is
\texttt{temperature}.

\section{Prototype}
\label{sec:basicprototype}

\subsection{Architecture}
\label{sub:basicprototypearchitecture}

The system architecture is an important part to provide a precise and reliable fall-analysis. 
Additionally, the aspect of patient compliance should be taken into consideration, because the 
system should be developed not only from the developer point of view, but it should be accepted 
by the patients. An important requirement of the elderly is that the hardware design should 
facilitate the freedom of movement. Furthermore, the system should guarantee a reliable 
functionality and a redundancy to protect the wearable system against a total system failure. 
With reference to these aspects a BAN in form of a belt was developed which 
includes a five sensor nodes which is based on ZigBee\footnote{\url{http://www.zigbee.org/}}.

\begin{figure}[!h]
  \centering
  \includegraphics[scale=0.25]{./img/belt}
  \caption[Fall-detection belt]{Fall-detection belt.}
  \label{fig:belt}
\end{figure}

Four of the sensor nodes are acting as end-devices (S1-S4) and the other node as a coordinator. 
The end-devices have attached an accelerometer and gyroscope which acquire continuously sensor 
data and is sent wirelessly to the coordinator using the ZigBee protocol. The coordinator receives 
the incoming data and it has the function to evaluate the patient's status. 
To test the system's accuracy different fall-types were reproduced with several test persons. A 
test procedure based on Li et al.~\cite{Li2009} and Pannurat et al.~\cite{Pannurat2014} was developed which includes 
several motions and fall-types which are typical in nursing homes and hospitals.

Taking into account the architecture of the prototype and the rule to define a fall (see Section \ref{subsec:analysis}), 
the EPL of EsperTech pattern to define the fall situation is the one shown in the Example~\ref{FallPattern}.

\begin{lstlisting}[basicstyle=\ttfamily\footnotesize,language=SQL,caption=Fall pattern,label=FallPattern]
  select a1.accelS1, a2.accelS1, a1.accelS2, 
   a2.accelS2 from 
   pattern [every(a1=BodyEvent(a1.accelS1 <= 9.81) -> 
   a2=BodyEvent(a2.accelS1 -a1.accelS1 >= 9.81 and 
   a1.PersonID = a2.PersonID) 
   where timer:within(1sec)) or every 
   (a1=BodyEvent(a1.accelS2 <= 9.81)
   -> a2=BodyEvent(a2.accelS2-a1.accelS2 >= 9.81
   and a1.PersonID = a2.PersonID) 
   where timer:within(1sec))];
 \end{lstlisting}

\textbf{*We should describe a little the EPL query*}
 
\subsection{Fall simulation test events} % \subsection{Experiment process}

The fall to generate the test events have been selected from~\cite{Li2009,Pannurat2014}. The fall consists on rolling into bed and fall (RBF).
In this study one person have been analysed; he has been doing for a period of 2 minutes the RBF fall. The analysed 
data and videos can be found in ~\cite{}. In the following lines the steps to simulate the fall with the generated events are described:

\subsubsection*{Study of the values} Given that the sensor 1 is the one that suffer the impact, its acceleration values are the first
to be analysed. The goal is to study the acceleration behaviour while a fall in order to generate test events, so the acceleration values are 
normalised ($N(m/s^2)$). After the normalisation the impacts of the falls, peaks, have to be detected; we have considered a peak when 
the normalised acceleration is greater than 0,7 ($N(m/s^2) > 0,7$). After applying the previous rule in all the fall data the impacts are detected.

While the data analysis, it has to be taken into account that the values suffer alterations because several factors: 
the person movement, the person bounces against something (floor, wall, etc), the collocation of the sensors to the 
original position after a fall, sensor pressure because an impact or the person is lying over it, etc.

 \begin{figure*}[!h]
  \includegraphics[scale=0.375]{Sensor1Sombras}
  \caption[Sensor 1 acceleration]{Sensor 1 acceleration.}
\end{figure*}

\subsubsection*{Fall identification and analysis} Once the peaks are detected, a range of values, including the peaks, are selected in order to 
analyse data properly. The range of extracted values are a set of data that happen is less than a time window of 1 second, to meet the fact described 
in~\cite{Luder2009}. The Table~\ref{tabla:Rango1} shows the acceleration value while the first RBF fall and the Table ~\ref{tabla:Rango2} shows the 
acceleration value while the second RBF fall.

  \begin{table}
    \centering
    \begin{minipage}[t]{.48\textwidth}
      \centering
      \begin{tabular}{*{5}{r}}
        \centering
\newcommand{\markop}[1]{#1\hfill}
\renewcommand{\arraystretch}{0.9}
\linespread{1}\selectfont
\begin{tabularx}{5.5cm}{@{}c|cc@{}}
  \toprule
  \multicolumn{1}{p{1.4cm}}{\centering \textsc{Tiempo} \\ ($ms$)} & \multicolumn{1}{p{1.25cm}}{\centering \textsc{Acel.} \\ ($m/s^2$)}& \multicolumn{1}{p{1.3cm}}{\centering \textsc{Acel. N} \\ ($N(m/s^2)$)} \\
  \midrule
0 & 6,29 & 0,02\\
10 & 7,57 & 0,05\\
20 & 20,7 & 0,33\\
30 & 24,07 & {\setlength{\fboxsep}{0pt}\colorbox{blue}{0,41}}\\
40 & 21,01 & 0,34\\
50 & 10,81 & 0,12\\
60 & 7,71 & 0,05\\
70 & 6,87 & 0,04\\
80 & 7,06 & 0,04\\
90 & 5,23 & {\setlength{\fboxsep}{0pt}\colorbox{bananayellow}{0,00}}\\
100 & 51,58 & {\setlength{\fboxsep}{0pt}\colorbox{applegreen}{1,00}}\\
110 & 11,01 & 0,12\\
120 & 12,12 & 0,15\\
130 & 11,77 & 0,14\\
140 & 11,26 & 0,13\\
150 & 11,16 & 0,13\\
  \bottomrule
\end{tabularx}
      \end{tabular}
      \caption{Acceleration - \textit{Fall 1}}%
      \label{tabla:Rango1}
    \end{minipage}
    \hfill
    \begin{minipage}[t]{.48\textwidth}
      \centering
      \begin{tabular}{*{5}{r}}
        \centering
\newcommand{\markop}[1]{#1\hfill}
\renewcommand{\arraystretch}{0.9}
\linespread{1}\selectfont
\begin{tabularx}{5.5cm}{@{}c|cc@{}}
  \toprule
  \multicolumn{1}{p{1.4cm}}{\centering \textsc{Tiempo} \\ ($ms$)} & \multicolumn{1}{p{1.25cm}}{\centering \textsc{Acel.} \\ ($m/s^2$)}& \multicolumn{1}{p{1.3cm}}{\centering \textsc{Acel. N} \\ ($N(m/s^2)$)} \\
  \midrule
0 & 5,98 & 0,08\\
10 & 6,31 & 0,08\\
20 & 8,2 & 0,13\\
30 & 9,21 & 0,16\\
40 & 19,92 & 0,43\\
50 & {\setlength{\fboxsep}{0pt}\colorbox{blue}{21,8}} & 0,48\\
60 & 16,52 & 0,34\\
70 & 14,41 & 0,29\\
80 & 9,97 & 0,18\\
90 & 6,54 & 0,09\\
100 & {\setlength{\fboxsep}{0pt}\colorbox{bananayellow}{3,01}} & 0,00\\
110 & {\setlength{\fboxsep}{0pt}\colorbox{applegreen}{42,34}} & 1,00\\
120 & 8,96 & 0,15\\
130 & 10,14 & 0,18\\
140 & 9,81 & 0,17\\
150 & 10,45 & 0,19\\
  \bottomrule
\end{tabularx}




      \end{tabular}
      \caption{Acceleration - \textit{Fall 2}}%
      \label{tabla:Rango2}
    \end{minipage}
  \end{table}

The first columns of the Tables represent the milliseconds ($ms$) when the acceleration was measured. The second
columns show the acceleration values ($m/s^2$) and in the third columns are deployed the normalised acceleration values 
($N(m/s^2)$) according to the fall maximum value. In Figure~\ref{fig:Sensor1} a comparison of the normalised acceleration behaviour 
while the previous RBF falls is shown. This comparison show a similar behaviour of the acceleration during the RBF falls.

\begin{figure}[!h]
  \centering
  \includegraphics[scale=0.5]{Comparativa}
  \caption[Acceleration comparative]{Acceleration comparative.}
  \label{fig:Sensor1}
\end{figure}

The acceleration behaviour during a RBF fall consists on:
\begin{enumerate}
 \item the variation of its values while the person is walking, the acceleration has a normalised values between [0, 0,35].
 \item the impact of the person's knees on the ground, the acceleration normalised values has to be greater than 0,7; as it was mentioned
 an impact happens when the normalised acceleration value is greater than 0,7.
 \item the normalised acceleration value decrease. The values of the normalised acceleration are between [0, 0,35]; the number of times that 
 the acceleration value is in the mentioned range is variable. That number of times depends on the high of the person; the higher person
 the longer range. Moreover, a subtle peak could appear as a consequence of a rebound.
 A subtle peak can be considered as a normalised acceleration value which is greater than 0,35 and less than 0,7; followed by a normalised
 acceleration value in the range [0, 0,35].
 \item a second impact happens when the person hit the ground, the acceleration normalised values has to be greater than 0,7.
 \item finally, the person is lying on the ground and the normalised acceleration value decrease. The values of the normalised acceleration are 
 between [0, 0,35] and no subtle peak appear. 
\end{enumerate}

The same analysis process has been done to the rest of sensors, and the behaviour of the acceleration of all of them follows the same pattern.
% Figure~\ref{fig:FFcomparison} shows the four sensors normalised acceleration behaviour while the previous FF fall of person 1, in the 
% Figure~\ref{} are represented the four sensors normalised acceleration behaviour while the previous FF fall of person 2. 
% The four sensors normalised acceleration behaviour while the previous FAW fall of person 1 are shown, and Figure~\ref{} the 
% four sensors normalised acceleration behaviour while the previous FAW fall of person 2 are represented.

\textbf{*We should include an EPL query to detect this RBF fall*}

\subsubsection*{To define the fall event} Once that the fall acceleration behaviour has been observed, the next step is to define the 
fall event in order to generate test events with IoT-TEG. As it was explained in Section~\ref{iotteg}, the event type attributes have
to be defined using the \texttt{<field>} element. The fall event contains one attribute, the acceleration, which is float, 
\texttt{type=``Float''}, and its values are not quoted, \texttt{quotes=``false''}. A new parameter in IoT-TEG has been defined as a 
consequence of the previous falls study. Given that the acceleration values follow a specific behaviour, it is necessary to include 
the \texttt{custom\_behaviour} property in the \texttt{<field>} element to define the behaviour of any event attribute; 
in this study, the acceleration. In the \texttt{custom\_behaviour} property the path to the file that includes the behaviour of the 
event attribute has to be written. The Example~\ref{FallEvent} shows the complete fall event definition (FallEventType).

\begin{lstlisting}[basicstyle=\ttfamily\footnotesize,language=XML,caption={Fall event type definition},label=FallEvent]
<?xml version="1.0" encoding="UTF-8"?>
<event name="FallEventType">
<block name="feeds" repeat="100">
 <field name="acceleration" quotes="false" type="Float" 
 custom_behaviour="/Path/To/Rule/File"></field>
</block>
</event>
\end{lstlisting}

IoT-TEG includes a new functionality, which has been implemented to simulate the desired behaviour of an 
event attribute with a \texttt{custom\_behaviour} property in its event type definition. This functionality 
allows to generate values of the event attribute following a behaviour that the user has described in a file.
In order to explain how the user has to define the desired behaviour of an event attribute, we are going
to use the RBF fall behaviour rules (see Example~\ref{RBFFallRules}). In a XML file the number of simulations has to be
indicated, the events involved in a simulation will be calculated according to the total number of events
to generate and the desired simulations. For example, if the number of test events to generate is 100, 
number indicated in the event type definition file, \texttt{repeat="100"}, and the number of desired 
simulations is 5, \texttt{simulations="5"}, number indicated in the behaviour rules definition file, 
the number of events involved to simulate the behaviour is 20. In the RBF fall example, the user ask to 
generate 100 test events and 5 falls (simulations), so 20 events will be used to define a RBF fall.

Variables can be defined if they are needed in the behaviour rules. They can be defined in the file where
the behaviour rules are included using the \texttt{<variables>} tags. To define them a name and a value 
have to be given to the variables. The value can be defined as a fixed value with the \texttt{value} 
property, or using a range with the \texttt{min} and \texttt{max} properties. Moreover, in some variables
are involved in the value of another variables; this is indicated using the variable with an specific 
format \texttt{\$(variable)}, see Example~\ref{RBFFallRules}. In addition, arithmetic operations can be done in the 
definition of the variable values, let see how using them in the RBF fall example. To define the acceleration behaviour
while a RBF fall three variables are defined: Roll, Fall and Impact. If we considered the maximum value of
the acceleration in the fall, the Roll variable is defined by a range where the acceleration value with...

\textbf{*TODO*}

\begin{lstlisting}[basicstyle=\ttfamily\footnotesize,language=XML,caption={Rules to define a RB fall},label=RBFFallRules]
  <?xml version="1.0" encoding="UTF-8"?>
  <custom_conditions simulations="5">
  <variables>
   <variable name="Impact" min="40.0" max="156.96"/>
   <variable name="Roll" min="$(Impact)/2-0.5" 
                             max="$(Impact)/2+0.5"/>
   <variable name="Fall" min="0.0" max="$(Roll)/5"/>
  </variables>
  <rules>
   <rule weight="0.25" min="$(Roll)/4" max="$(Roll)" 
                                    sequence="inc"/>
   <rule weight="0.25" min="$(Roll)" max="$(Fall)" 
                                    sequence="dec"/>
   <rule weight="1" value="$(Impact)"/>
   <rule weight="0" min="$(Roll)/2-0.25" 
                              max="$(Roll)/2+0.25"/>
  </rules>
  </custom_conditions>
\end{lstlisting}

\subsection{Founded problems}

\textbf{*We should talk about synchronisation, the person have to wait after falling and the duration of the battery*}

\section{Up-grated prototype}
\label{sec:improvedprototype}

\subsection{Architecture}
\label{sub:improvedprototypearchitecture}

\textbf{*To talk about the improvements in the prototype*}

\subsection{Fall simulation test events} % \subsection{Experiment process}

The second fall to generate the test events consists on the impact of the person with a wall and falling using the knees and then
using the chest, \textit{fall against wall} (FAW). In this study two persons have been analysed; they have been doing for a period 
of 2 minutes the FAW fall. The analysed data and videos can be found in ~\cite{}.

In this analysis the same steps that the ones described in Section~\ref{sec:basicprototype} have been done:

\subsubsection*{Study of the values} Given that the sensor 1 is the one that suffer the impact, its acceleration values are the first
to be analysed. The acceleration values have been normalised ($N(m/s^2)$). After the normalisation the impacts of the falls, peaks, 
have to be detected ($N(m/s^2) > 0,7$). After applying the previous rule in all the fall data and taking into account the alterations 
because the mentioned factors, the impacts are detected.

\subsubsection*{Fall identification and analysis} Once the peaks are detected, a range of values, including the peaks, 
are selected in order to analyse data properly. The range of extracted values are a set of data that happen is less 
than a time window of 1 second, to meet the fact described in~\cite{Luder2009}. The Table~\ref{tabla:FAW} shows the 
acceleration value while one FAW fall of person 1 and person 2. 

\begin{table}
 \centering
 \begin{tabular}{*{5}{r}}
   \input{table-FAW}
 \end{tabular}
 \caption{FAW fall acceleration, person 1 and 2}%
 \label{tabla:FAW}
\end{table}

The first and fourth columns of the Table~\ref{tabla:FAW} represent the seconds and milliseconds ($s.ms$) when the acceleration 
was measured. The second and fifth columns show the acceleration values ($m/s^2$) and in the third and sixth columns are deployed 
the normalised acceleration values ($N(m/s^2)$) according to the fall maximum value. In Figure~\ref{fig:FAWcomparison} a comparison 
of the normalised acceleration behaviour while the previous FAW falls of person 1 and person 2 is shown. These comparisons show a 
similar behaviour of the acceleration during the FAW fall.

\begin{figure}
  \centering
  \includegraphics[scale=0.4]{img/FAWComparison.png}
  \caption[Acceleration during FAW fall]{Acceleration comparison during FAW fall.}
  \label{fig:FAWcomparison}
\end{figure}

The acceleration behaviour during the FAW consists on:
\begin{enumerate}
 \item the variation of its values while the person is walking, the acceleration has a normalised values between [0, 0,35].
 \item as a consequence of the impact of the person against a wall, the acceleration normalised values has to be greater than 0,7.
 \item the normalised acceleration value decrease. The values of the normalised acceleration are between [0, 0,35]; the number of times that 
 the acceleration value is in the mentioned range is variable. That number of times depends on the high of the person; the higher person
 the longer range and if the person retains the fall thanks to the wall. Moreover, a subtle peak could appear as a consequence of a rebound.
 \item a second impact happens when the person hit the ground, the acceleration normalised values has to be greater than 0,7.
 \item finally, the person is lying on the ground and the normalised acceleration value decrease. The values of the normalised acceleration are 
 between [0, 0,35] and no subtle peak appear. 
\end{enumerate}

The same analysis process has been done to the rest of sensors, and the behaviour of the acceleration of all of them follows the same pattern.

\subsubsection*{To define the fall event} Once that the fall acceleration behaviour has been observed, the next step is to define the 
fall event in order to generate test events with IoT-TEG. Given that the involved event attribute in this fall is the acceleration, 
the Example~\ref{FallEvent} in Section~\ref{sec:basicprototype} can be used to define the fall event (FallEventType). The rules to
define the behaviour of the acceleration in this type of fall is shown in Example~\ref{FAWFallRules}.

\begin{lstlisting}[basicstyle=\ttfamily\footnotesize,language=XML,caption={Rules to define a FAW fall},label=FAWFallRules]
  <?xml version="1.0" encoding="UTF-8"?>
  <custom_conditions simulations="5">
  <variables>
   <variable name="Base" min="0.0" max="500.0"/>
   <variable name="Impact" min="$(Base)*0.7" 
                             max="$(Impact)/2+0.5"/>
   <variable name="Fall" min="0.0" max="$(Roll)/5"/>
  </variables>
  <rules>
   <rule weight="0.25" min="$(Roll)/4" max="$(Roll)" 
                                    sequence="inc"/>
   <rule weight="0.25" min="$(Roll)" max="$(Fall)" 
                                    sequence="dec"/>
   <rule weight="1" value="$(Impact)"/>
   <rule weight="0" min="$(Roll)/2-0.25" 
                              max="$(Roll)/2+0.25"/>
  </rules>
  </custom_conditions>
\end{lstlisting}

\textbf{* Like with the RBF fall, we should include a new fall definition of FAW type of fall.*}

\textbf{* We must compare the prototype with and without the improvements*}

\section{Conclusion and Future Work}
\label{sec:conclusions}

\textbf{*We talk about new functionalities in IoT-TEG related with the future work in the fall detection prototype.*}

\section*{Acknowledgment}

Paper partially funded by The Ministry of Economy and Competitiveness (Spain) and the FEDER Fund, under the National Program for 
Research, Development and Innovation, Societal Challenges Oriented, Project DArDOS TIN2015-65845-C3-3-R, and the
Programa de Fomento e Impulso de la actividad Investigadora of the University of Cádiz.

\bibliographystyle{IEEEtran}
% argument is your BibTeX string definitions and bibliography database(s)
\bibliography{references}

% that's all folks
\end{document}


